\documentclass[UTF8,lualatex]{ctexbeamer}
\usepackage{tabu}
\usepackage{amsmath}
\usepackage{graphicx}
\usepackage{minted}

\usetheme{Frankfurt}
\setbeamertemplate{note page}[plain]
% $BEAMER_NOTES

\title{\kaishu Modern C++ and Beyond\\
    第二课:Move Semantics}
    \author{陶大}
    \date{2022年7月}

\newminted{cpp}{autogobble}
\newminted{rust}{autogobble}

\renewcommand{\emph}[1]{\textbf{#1}}

\begin{document}
\songti

\begin{frame}[plain]
    \titlepage
\end{frame}

\begin{frame}[plain]
    \begin{center}
        \includegraphics[height=.8\textheight]{pics/kungfu_category.jpg}
    \end{center}
\end{frame}

\begin{frame}[plain]
    \tableofcontents
\end{frame}

\section{Why move-semantics is useful}
\subsection{Return-Value Optimization}

\begin{frame}[fragile]
    \frametitle{Return-Value Optimization, RVO}
    \begin{block}{An example from zlib}
        \begin{cppcode}
            int deflateInit(z_streamp strm, int level);
            int deflate(z_streamp strm, int flush);
            int deflateEnd(z_streamp strm);
        \end{cppcode}
    \end{block}
    \begin{alertblock}{Question}
        为什么返回值是int,而不是一个struct?
    \end{alertblock}
\end{frame}

\begin{frame}[fragile]
    \frametitle{Return-Value Optimization, RVO}
    \begin{columns}
        \begin{column}{.45\textwidth}
            \begin{onlyenv}<1>
                \includegraphics{pics/stack0.png}
            \end{onlyenv}
            \begin{onlyenv}<2>
                \includegraphics{pics/stack1.png}
            \end{onlyenv}
        \end{column}
        \begin{column}{.45\textwidth}
            \begin{cppcode}
                int g(int x) {
                    int y = x * x;
                    return y;
                }
                void f() {
                    int a = g(1);
                }
            \end{cppcode}
        \end{column}
    \end{columns}
\end{frame}

\note[itemize]{
\item 左图是右边程序的一个可能的栈内存的排布。
    \begin{itemize}
    \item \emph{此处应询问听众是否了解栈内存和堆内存!}
    \item \mintinline{cpp}{return y} 将y复制到g的返回值。
        然后弹掉g的栈。
    \end{itemize}
    这个复制是可以被优化掉的。
    做法是直接将y指向返回值。
}

\begin{frame}[fragile]
    \frametitle{Return-Value Optimization, RVO}
    \begin{columns}[b]
        \begin{column}{.45\textwidth}
            \includegraphics{pics/stack0.png}
        \end{column}
        \begin{column}{.45\textwidth}
            \includegraphics{pics/stack1.png}
        \end{column}
    \end{columns}
    \begin{exampleblock}{~}
        \begin{cppcode}
            T g(T x) {
                T y = do_something(x);
                return y;
            }
            void f() {
                T a = g(...);
            }
        \end{cppcode}
    \end{exampleblock}
\end{frame}

\note[itemize]{
\item 但如果我们把int换成某个类的话,事情就复杂了。
    在C++里,复制一个类的对象,意味着调用它的拷贝构造函数。
    那么,
    \begin{itemize}
    \item 有些类的复制极为耗费时间,比如vector这样的collections。
    \item 有些类本质上不可复制,比如mutex。
    \end{itemize}
    这就意味着,一个优化有外部可观察的区别,那就不再是一个优化。
    而是一个语言的feature。
}

\subsection{XValue}

\begin{frame}[fragile]
    \frametitle{XValue}
    \begin{exampleblock}{~}
        \begin{cppcode}
            void f() {
                T a = g(...);
            }
        \end{cppcode}
    \end{exampleblock}
\end{frame}

\note[itemize]{
\item 刚才的例子里还有一处赋值,就是这里。
    这里赋值号右边的对象在C++里称为xvalue。
    也有一些早期介绍move semantics的文章称为临时值。
    其实这个值一点都不临时,生命期和局部值一样长。
    它只是匿名。
}

\begin{frame}[fragile]
    \frametitle{XValue}
    \begin{exampleblock}{有名万物之母}
        \begin{cppcode}
            void f() {
                const T& a = g(...);
            }
        \end{cppcode}
    \end{exampleblock}
\end{frame}

\note[itemize]{
\item 那么直接使用这个对象可以吗?
    可以的。
    通过const引用可以给它一个名字,就可以用了。
\item 这里其实藏着C++设计的一个问题:
    为什么不可以mutable引用?
    我其实不理解这个不一致的设计。
}

\subsection{Collections}

\begin{frame}[fragile]
    \frametitle{Collections}
    \begin{exampleblock}{~}
        \begin{cppcode}
            std::vector<T> ts;
            const T& x = gen_t();
            ts.push_back(x);
            const T& y = gen_t();
            ts.push_back(y);
        \end{cppcode}
    \end{exampleblock}
    \begin{block}{Question}
        \begin{itemize}
        \item x被拷贝(构造)了几遍?
        \end{itemize}
    \end{block}
\end{frame}

\note[itemize]{
\item 答案是:不知道。
    \begin{itemize}
    \item 在第一次入库的时候肯定会拷贝一次。
    \item 问题是,当vector扩张的时候,会不会再拷贝一次呢?
    \end{itemize}
}

\begin{frame}
    \frametitle{Collections}
    \begin{alertblock}{~}
        一条规则:应当避免使用std::vector。代之以std::deque。
    \end{alertblock}
\end{frame}

\note[itemize]{
\item 关于这条规则,以后再展开。
    std::deque的不扩张保证也可以避免指针失效。
    所以在有move semantics的现代C++中也仍然适用。
}

\begin{frame}[fragile]
    \frametitle{Collections}
    \footnotesize
    \begin{columns}
        \begin{column}{.45\textwidth}
            \begin{exampleblock}{~}
                \begin{cppcode}
                    std::deque<T> ts;
                    const T& x = gen_t();
                    ts.push_back(x);
                \end{cppcode}
            \end{exampleblock}
        \end{column}
        \begin{column}{.45\textwidth}
            \begin{exampleblock}{~}
                \begin{cppcode}
                    std::deque<T> ts;
                    ts.push_back(T());
                    gen_t(ts.back());
                \end{cppcode}
            \end{exampleblock}
        \end{column}
    \end{columns}
\end{frame}

\note[itemize]{
\item 回到第一次拷贝。
    我们知道,有些对象不太好拷贝。
    为此在C++11之前,老人家往往会这么设计API。
    非常丑,但是实用。
    \begin{itemize}
    \item 反直觉。
        默认构造的对象必须非常轻量。
        而这件事情不见得所有类都有自然且轻量的默认值。
        自然意味着类似自然数里的“0”。
        还要求“0”的实现轻量,强人所难。
        比如std::vector。
        只要malloc一片内存,就不能算轻量。
    \item 鼓励mutable对象的使用。
    \end{itemize}
}

\section{C++: Move Semantics and Consequences}

\begin{frame}[fragile]
    \frametitle{C++: Move Semantics}
    \begin{block}{以上场景的共性}
        \emph{所有权的转移}
        \begin{itemize}
        \item 所有权、对象生命期是同一个问题的一体两面。
        \item 所有权问题并非C++一家的问题。
        \end{itemize}
    \end{block}
    \scriptsize
    \begin{columns}[t]
        \begin{column}{.65\textwidth}
            \begin{exampleblock}{try-with-resource in Java}
                \begin{minted}[autogobble]{java}
                    try (
                      FileReader fr = new FileReader(path);
                      BufferedReader br = new BufferedReader(fr))
                    {
                      return br.readLine();
                    }
                \end{minted}
            \end{exampleblock}
        \end{column}
        \begin{column}{.35\textwidth}
            \begin{exampleblock}{defer in Go}
                \begin{minted}[autogobble]{go}
                    mutex.Lock()
                    defer mutex.Unlock()
                    count++
                \end{minted}
            \end{exampleblock}
        \end{column}
    \end{columns}
\end{frame}

\note[itemize]{
\item 在对象生命期的终点,谁(一个对象或一个函数)管杀管埋,谁就有所有权。
    RAII只是所有权的一种实现。
\item 所有权问题并非C++一家的问题。
    在Java/Go这些gc语言里,同样有这个问题。
    千万不要认为JVM或Go VM持有对象的所有权。
    GC只负责复用死对象的内存。
    对象占用的资源不只是内存,更准确地说,不只是堆内内存。
    文件句柄、mutex、堆外内存(off-heap memory,不被GC管理的内存)都是。
    所以才有Java的try-with-resource statement和Go的defer statement。
}

\begin{frame}[fragile]
    \frametitle{C++: Move Semantics}
    \begin{block}{API side}
        \begin{cppcode}
            struct T {
                T(T&&); // move ctor
                T& operator=(T&&); // move assign
            };
        \end{cppcode}
    \end{block}
    \begin{block}{Usage}
        \begin{cppcode}
            T a = ...;
            T x(std::move(a));
            T b = ...;
            T y;
            y = std::move(b);
        \end{cppcode}
    \end{block}
\end{frame}

\note[itemize]{
\item C++ std规定了什么时候应用move semantics。
    太复杂了,大家自己看书去。
    我自己是记不住的,一律std::move。
    顺便也在使用处留一个visual marker。
}

\subsection{RVO: Revisited}

\begin{frame}[fragile]
    \frametitle{RVO: revisited}
    \begin{cppcode}
        T g(T x) {
            T y = do_something(x);
            return y;
        }
    \end{cppcode}
\end{frame}

\note[itemize]{
\item C++ std规定,编译器实现可以做RVO,也可以将y move到return value。
}

\begin{frame}[fragile]
    \frametitle{RVO: revisited}
    \begin{cppcode}
        T g(T x) {
            T y;
            T z;
            while(...) {
                // do something to y
                // do other things to z
                if (...) {
                    return y;
                } else {
                    return z;
                }
            }
        }
    \end{cppcode}
\end{frame}

\note[itemize]{
\item 像这种情况,编译器比较难判断是否可以RVO。
    那么遵照标准,编译器可以实现成在return处move。
}

\subsection{Rules}

\begin{frame}
    \frametitle{C++: Move Semantics}
    \begin{alertblock}{rules}
    \begin{itemize}
        \item 移动之后的对象只能析构,不能做任何事情!
        \item 鼓励所有的类都实现move semantics。
        \item 只要涉及所有权的转移,接收方只接受对象。
            不要引用,不要裸指针。
            反之亦然。
    \end{itemize}
    \end{alertblock}
\end{frame}

\note[itemize]{
\item 智能指针是对象。
    智能指针虽然有“指针”之名,其实和裸指针有语义上的不同。
    \begin{itemize}
    \item 裸指针:我指向一片内存,那里有个对象。
    \item 智能指针:堆上有个对象,可以从我这里取用。
    \end{itemize}
\item 第一条规则其实很难做到。
    C++标准里补充规定了标准库的实现,在move之后回到默认构造的状态。
    这一条充分说明了C++程序员有多不靠谱。
    所以留下visual marker非常重要。
\item 第二条规则一般不难做到,但很容易忘记。
    我自己也经常忘。
\item 那么,Rust怎么处理这两条规则的呢?
}

\section{Move Semantics in Rust}

\begin{frame}[fragile]
    \frametitle{Move Semantics in Rust}
    \begin{block}{~}
        \begin{rustcode}
            fn g<T>(x: T) -> T {
                // do something on `x`
            }

            fn f() {
                let a = g(T::default());
                let b = a;
                // `a` 一般不能再用
            }
        \end{rustcode}
    \end{block}
\end{frame}

\note[itemize]{
\item 编译器禁止move后使用。
\item move=memcpy。
    编译器给出默认move ctor/assignment。
    不需要用户自己实现,因此也就不会忘实现。
    本来无一物,何处染尘埃。
}

\begin{frame}[fragile]
    \frametitle{Move Semantics in Rust}
    \begin{block}{Self-referential objects}
        \begin{rustcode}
            struct T {
                value: i64,
                ptr: *const i64, // refer to `self::value`
            }
        \end{rustcode}
    \end{block}
\end{frame}

\note[itemize]{
\item 当一个对象被move(memcpy到另一片内存)后,value没问题,但ptr指向老的内存。
    为了避免由此引入的unsafety,程序员需要告诉编译器:这个类的对象统统不允许move。
\item 实际上,这类对象一般在堆上分配。
    用的时候都会裹上一个智能指针。
    而这个智能指针是可以move的。
    也就是说,库作者注意注意就可以了,用户不会有什么不适。
\item 我个人更喜爱rust的设计。
    C++延续了copy ctor/assign的一贯设计。
    给我的感觉是,解决了一个问题,好像又没完全解决。
}

\begin{frame}[plain]
    \begin{center}
        \includegraphics{pics/secret_kungfu.jpg}
    \end{center}
\end{frame}

\end{document}
