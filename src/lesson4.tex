\documentclass[UTF8,lualatex]{ctexbeamer}
\usepackage{tabu}
\usepackage{amsmath}
\usepackage{graphicx}
\usepackage{minted}

\usetheme{Frankfurt}
\setbeamertemplate{note page}[plain]
% $BEAMER_NOTES

\title{\kaishu Modern C++ and Beyond\\
    第四课:并行计算模型、线程与线程池}
    \author{陶大}
    \date{2022年8月}

\newminted{cpp}{autogobble}

\renewcommand{\emph}[1]{\textbf{#1}}

\begin{document}
\songti

\begin{frame}[plain]
    \titlepage
\end{frame}

\begin{frame}[plain]
    \begin{center}
        \includegraphics[height=.8\textheight]{pics/kungfu_category.jpg}
    \end{center}
\end{frame}

\begin{frame}[plain]
    \tableofcontents
\end{frame}

\section{Something happens concurrently}

\subsection{Physical Thread}

\begin{frame}[fragile]
    \frametitle{What is thread?}
    \begin{columns}[t]
        \begin{column}{.5\textwidth}
            \begin{block}{Logical}
                \begin{itemize}
                    \item Something happens concurrently.
                    \item Something is required in the future.
                \end{itemize}
            \end{block}
        \end{column}
        \begin{column}{.5\textwidth}
            \begin{block}{Physical}
                \begin{itemize}
                    \item A set of kernel-scheduled entities share a same memory space.
                \end{itemize}
            \end{block}
        \end{column}
    \end{columns}
    \begin{exampleblock}{std::thread}
        \scriptsize
        \begin{cppcode}
            #include <thread>

            std::thread j1([](){
                std::cout << "haha" << std::endl;
            });
            j1.join();
        \end{cppcode}
    \end{exampleblock}
\end{frame}

\note[itemize]{
\item 今天这一讲,第一个主题就是,我们在讲线程时,究竟在谈论什么。
\item 关于线程,有三个既相互独立又紧密相关的观念。
    其中physical thread可以视作concurrency观念的某种实现。
}

\begin{frame}
    \frametitle{Physical Thread}
    \begin{center}
        \includegraphics[height=.8\textheight]{pics/virtual_mem.png}
    \end{center}
\end{frame}

\note[itemize]{
\item 经典的虚拟内存布局。
    需要指出的是,线程栈eager分配内存。
    也就是说,有一个线程就有固定大小的物理内存被占掉。
    线程栈的大小和具体的系统有关。
    linux平台上可以通过ulimit查看。
    一般默认8MB。
\item 所以,有点什么事情就起个物理线程去做是很扯淡的。
    主要是内存吃不消。
    我在阿里云见过4K多线程,直接32GB内存就没了,还什么事情都没做呢。
}

\subsection{Thread Pool and Tasks}

\begin{frame}[fragile]
    \frametitle{Thread Pool and Tasks}
    \scriptsize
    \begin{columns}
        \begin{column}{.5\textwidth}
            \begin{exampleblock}{physical thread}
                \begin{cppcode}
                    std::atomic<int> s_cnt;
                    std::thread j1([](){
                        s_cnt.fetch_add(1);
                    });
                    j1.join();
                    assert(s_cnt.load() == 1);
                \end{cppcode}
            \end{exampleblock}
        \end{column}
        \begin{column}{.5\textwidth}
            \begin{exampleblock}{thread pool}
                \begin{cppcode}
                    noodles::Noodles tp(8, 200);
                    std::atomic<int> s_cnt;
                    auto res = tp.submit([]() -> void {
                        s_cnt.fetch_add(1);
                    });
                    res.wait();
                    assert(s_cnt.load() == 1);
                \end{cppcode}
            \end{exampleblock}
        \end{column}
    \end{columns}
\end{frame}

\note[itemize]{
\item 既然起物理线程代价有点大,很自然地,池化。
    也就是说,把concurrently execution拆分成一个一个的tasks,然后提交给线程池。
    把计算与其执行分开。
\item 在这个简单case上,看起来和原先的用法没有体验上的区别。
}

\begin{frame}[fragile]
    \frametitle{Thread Pool and Tasks}
    \begin{columns}
        \scriptsize
        \begin{column}{.5\textwidth}
            \begin{exampleblock}{physical thread}
                \begin{cppcode}
                    std::thread j1([](){
                        ofstream f1("f1.txt");
                        f1 << "haha" << endl;
                    });
                    j1.join();
                \end{cppcode}
            \end{exampleblock}
        \end{column}
        \begin{column}{.5\textwidth}
            \begin{exampleblock}{thread pool}
                \begin{cppcode}
                    noodles::Noodles tp(8, 200);
                    auto res = tp.submit([]() -> void {
                        ofstream f1("f1.txt");
                        f1 << "haha" << endl;
                    });
                    res.wait();
                \end{cppcode}
            \end{exampleblock}
        \end{column}
    \end{columns}
    \begin{alertblock}<2>{~}
        不要在线程池里做阻塞操作!
    \end{alertblock}
\end{frame}

\note[itemize]{
\item 在这个例子里,线程池的使用有点问题:阻塞地做IO。
    线程池里的线程数量有限,用一根少一根。
    如果这么搞,很快就没线程跑任务了,而外面看起来CPU并不高。
}

\begin{frame}[fragile]
    \frametitle{Thread Pool and Tasks}
    \begin{columns}[t]
        \scriptsize
        \begin{column}{.5\textwidth}
            \begin{exampleblock}{bad example}
                \begin{cppcode}
                    mutex mtx;
                    noodles::Noodles tp(8, 200);
                    auto res = tp.submit([]() -> void {
                        lock_guard<mutex> g(mtx);
                        // do something
                    });
                    res.wait();
                \end{cppcode}
            \end{exampleblock}
        \end{column}
        \begin{column}{.5\textwidth}
            \begin{exampleblock}{not-so-bad example}
                \begin{cppcode}
                    mutex mtx;

                    void f(noodles::Noodles& tp) {
                        if (mtx.try_lock()) {
                            lock_guard<mutex> lock(
                                mtx, adopt_lock);
                            // do something
                        } else {
                            tp.submit([&]() {
                                f(tp);
                            })
                        }
                    }

                    noodles::Noodles tp(8, 200);
                    tp.submit([&]() {
                        f(tp);
                    });
                \end{cppcode}
            \end{exampleblock}
        \end{column}
    \end{columns}
\end{frame}

\note[itemize]{
\item 右边的要点是检测资源的readiness。
    若不就绪,则将自身再次提交给线程池。
    由此便不会阻塞住工作线程。
\item 但是整个事情还是不对。
    检测资源的readiness既boring,又浪费CPU。
    这个问题的根源,对比物理线程,在于线程池只对应计算,OS调度的那块没有对应物。
}

\subsection{Coroutine}

\begin{frame}
    \frametitle{Coroutine}
    \begin{block}{~}
        \begin{itemize}
            \item 又名M:N thread, green thread, fiber
            \item co代表cooperative
        \end{itemize}
    \end{block}
\end{frame}

\note[itemize]{
\item 这里cooperative的意思是,执行流一直在一个logical thread上。
    直到该logical thread \emph{主动} suspend到一个资源上去。
}

\begin{frame}[fragile]
    \frametitle{Coroutine}
    \begin{columns}[t]
        \scriptsize
        \begin{column}{.5\textwidth}
            \begin{exampleblock}{physical thread}
                \begin{cppcode}
                    mutex mtx;
                    thread t1([]() -> void {
                        lock_guard<mutex> g(mtx);
                        // do something
                    });
                \end{cppcode}
            \end{exampleblock}
        \end{column}
        \begin{column}{.5\textwidth}
            \begin{exampleblock}{coroutine}
                \begin{cppcode}
                    cppcoro::async_mutex mtx;

                    cppcoro::task<> f() {
                        auto lock = co_await
                            mtx.scoped_lock_async();
                        // do something
                    }
                \end{cppcode}
            \end{exampleblock}
        \end{column}
    \end{columns}
\end{frame}

\note[itemize]{
\item co\_await会将执行流切换到userspace scheduler。
    由userspace scheduler做资源readiness检测。
    当资源就绪,scheduler再把执行流切换回来。
\item C++20 coroutine生态还非常原始。
    比如async\_mutex的实现是TTAS。
    相比Rust的tokio,简直就是那啥。
}

\begin{frame}
    \frametitle{Coroutine}
    \begin{block}{有栈 vs.\ 无栈}
        \begin{itemize}
            \item 有栈:兼容性好,切换代价比物理线程稍低
            \item 无栈:性能最好,内存最省
        \end{itemize}
    \end{block}
\end{frame}

\note[itemize]{
\item C++20之前的协程库基本都站有栈这边。
    C++20之后的协程库、Rust, C\#, Python, JS都站无栈这边。
    Golang比较鸡贼,它搞了一个 segmented stack。
\item 有栈协程
    \begin{itemize}
        \item 因为它们有栈,因此理论上各种库(除了调用阻塞系统调用)、各种工具(perf/monitoring之类)都没什么问题。
        \item 一般它们的栈是lazy分配,所以能解决一启动就吃大量物理内存的问题。
            但是一旦一根协程碰过的内存是不会被其他人复用的。
            即使后面缩回去了,也不会还给系统(这和gc还不一样)。
            当然,有栈协程针对的场景是大量短命协程。
            上述问题在这个场景里不严重。
    \end{itemize}
\item 无栈协程本质上是一个自动机。
    \begin{itemize}
        \item 一般需要在语言层面引入新的关键词。
            因此,兼容性差些。
        \item 也有不动语言的实现,比如tokio 0.x。
            但是需要用户有更强的函数式编程的背景。
    \end{itemize}
}

\subsection{Actor}

\begin{frame}[fragile]
    \frametitle{Actor}
    \begin{block}{Actor}
        每根协程搭配一个无限长队列
    \end{block}
    \begin{exampleblock}{Erlang Process}
        \scriptsize
        \begin{minted}[autogobble,linenos]{erlang}
            loop() ->
                receive
                    {sys, Msg} ->
                        handle_sys_msg(Msg),
                        loop();
                    {From, Msg} ->
                        Reply = handle_msg(Msg),
                        From ! Reply,
                        loop()
            end.
        \end{minted}
    \end{exampleblock}
\end{frame}

\note[itemize]{
\item 今天工程上实际有人用的Actor model,著名的只有两个Erlang和Java Akka。
    其中Erlang资格最老,设计也最干净。
\item Erlang process名为process其实是协程。
    每个Erlang process都配有一个无限长的队列,称为mailbox。
    \begin{itemize}
        \item 示例代码第3行的receive,是从自己的mailbox里拿一个message,然后处理。
        \item 第8行的 \mintinline{erlang}|From ! Reply| 是向From所指的进程的mailbox里追加一条Reply所指的消息。
    \end{itemize}
\item Actor model最大的价值是,它其实是一个建模工具。
    业务上会有大量的并发agent。
    我们人类做工作就是大量个体依照制度、流程的约束并发地处理各自的任务。
    作为建模工具,其原则是
    \begin{center}
        \emph{Communicate, not share!}
    \end{center}
\item 一个小练习: 如何用actor model实现mutex?
\item Actor model在实践中有两个问题:
    \begin{itemize}
        \item 小问题:mailbox缺少流控。
            内存容易打爆。
        \item 大问题:整个系统完全都是分布式系统。
            分布式系统开发难度更大,对开发者的技能要求更高。
            虽然在现代系统架构里,分布式系统基本不可避免。
    \end{itemize}
}

\subsection{CSP}

\begin{frame}[fragile]
    \frametitle{Communicating Sequential Process, CSP}
    \begin{block}{Golang}
        \scriptsize
        \begin{minted}[autogobble,linenos]{golang}
            func fibonacci(c, quit chan int) {
                x, y := 0, 1
                for {
                    select {
                    case c <- x:
                        x, y = y, x+y
                    case <-quit:
                        return
                    }
                }
            }
            func main() {
                c := make(chan int, 3)
                quit := make(chan int, 1)
                go func() {
                    for i := 0; i < 10; i++ {fmt.Println(<-c)}
                    quit <- 0
                }()
                fibonacci(c, quit)
            }
        \end{minted}
    \end{block}
\end{frame}

\note[itemize]{
\item Go是CSP的工业化语言的代表。
\item 对开发者来讲,最重要的是,CSP和Actor model的哲学是一样的
    \begin{center}
        \emph{Communicate, not share!}
    \end{center}
    差异有两点比较重要:
    \begin{itemize}
        \item 有界队列(channel)
        \item 单进程多队列
    \end{itemize}
}

\section{Something is required in the future}

\subsection{Future/Promise}

\begin{frame}[fragile]
    \frametitle{Future/Promise}
    \scriptsize
    \begin{columns}[t]
        \begin{column}{.5\textwidth}
            \begin{exampleblock}{future from a promise}
                \begin{cppcode}
                    std::promise<int> p;
                    std::future<int> f = p.get_future();
                    std::thread t(
                        [&p]{ p.set_value(42); });
                    t.join();
                \end{cppcode}
            \end{exampleblock}
            \begin{exampleblock}{future from an async()}
                \begin{cppcode}
                    std::future<int> f = std::async(
                        []{ return 42; });
                \end{cppcode}
            \end{exampleblock}
        \end{column}
        \begin{column}{.5\textwidth}
            \begin{exampleblock}{future from a packaged task}
                \begin{cppcode}
                    std::packaged_task<int()> task(
                        []{ return 42; });
                    std::future<int> f = task.get_future();
                    std::thread t(std::move(task));
                    j.join();
                \end{cppcode}
            \end{exampleblock}
        \end{column}
    \end{columns}
    \begin{block}{when the value is required}
        \begin{cppcode}
            f.wait();
            std::cout << f.get() << std::endl;
        \end{cppcode}
    \end{block}
\end{frame}

\begin{frame}[fragile]
    \frametitle{Future/Promise}
    \scriptsize
    \begin{columns}[t]
        \begin{column}{.5\textwidth}
            \begin{exampleblock}{future from an async()}
                \begin{cppcode}
                    future<int> f = std::async(
                        launch::async | launch::deferred
                        []{ return 42; });
                \end{cppcode}
            \end{exampleblock}
        \end{column}
        \begin{column}{.5\textwidth}
            \begin{block}{when the value is required}
                \begin{cppcode}
                    f.wait();
                    std::cout << f.get() << std::endl;
                \end{cppcode}
            \end{block}
        \end{column}
    \end{columns}
\end{frame}

\note[itemize]{
\item Future,顾名思义,是未来要用的值。
    \begin{itemize}
        \item 在其依赖就绪的时候提交计算
        \item 在需用的时候等待其值就绪
    \end{itemize}
    细究下来,有两种场景: async evalution 和 lazy evaluation,
    分别对应lauch::async和lauch::deferred。
}

\begin{frame}[fragile]
    \frametitle{作为抽象工具的lazy eval}
    \scriptsize
    \begin{columns}[t]
        \begin{column}{.5\textwidth}
            \begin{block}{~}
                \begin{cppcode}
                    auto start0 =
                        chrono::steady_clock::now();
                    // do some heavy things
                    int x0 = calc0(...);
                    auto end0 =
                        chrono::steady_clock::now();
                    cout << "get " << x0
                        << " in " << (end0 - start0)
                        << endl;

                    auto start1 =
                        chrono::steady_clock::now();
                    // do some other heavy things
                    int x1 = calc1(...);
                    auto end1 =
                        chrono::steady_clock::now();
                    cout << "get " << x1
                        << " in " << (end1 - start1)
                        << endl;
                \end{cppcode}
            \end{block}
        \end{column}
        \begin{column}{.5\textwidth}
            \begin{block}<2>{~}
                \begin{cppcode}
                    void bench(future<int>& fut) {
                        auto start =
                            chrono::steady_clock::now();
                        fut.wait();
                        auto end =
                            chrono::steady_clock::now();
                        cout << "get " << fut.get()
                            << " in " << (end - start)
                            << endl;
                    }

                    future<int> f0 = std::async(
                        std::lauch::deferred,
                        [&](){return calc0(...);});
                    bench(f0);

                    future<int> f1 = std::async(
                        std::lauch::deferred,
                        [&](){return calc1(...);});
                    bench(f1);
                \end{cppcode}
            \end{block}
        \end{column}
    \end{columns}
\end{frame}

\note[itemize]{
\item 此处应暂停一下。令听众思考如何在左边代码上减少重复。
\item Lazy eval要求其值必定在需用的时候才在使用者线程上计算。
    反正也要在自己线程上计算,为何要如此复杂呢?
    因为这是一个抽象工具。
    它隔绝了计算一个值所需用的依赖和使用这个值的场景,
    即,
    使用的地方只需要关心值本身,而不需要关心如何计算。
\item \emph{可以这么用,不等于应该这么用。}
}

\begin{frame}[fragile]
    \frametitle{Async eval}
    \begin{block}{~}
        \scriptsize
        \begin{cppcode}
            void bench(future<int>& fut) {
                auto start = chrono::steady_clock::now();
                fut.wait();
                auto end = chrono::steady_clock::now();
                cout << "get " << fut.get()
                    << " in " << (end - start)
                    << endl;
            }

            future<int> f0 = std::async(
                std::lauch::async,
                [&](){return calc0(...);});
            bench(f0);
        \end{cppcode}
    \end{block}
\end{frame}

\note[itemize]{
\item 这个程序和前面的区别只有 \mintinline{cpp}|std::lauch::deferred| 换成了 \mintinline{cpp}|std::launch::async| 。
    但是结果是不符合 \mintinline{cpp}|bench()| 的预期的。
    有可能因为抢跑而显得很快。
    也有可能因为启动并行的overhead导致更慢。
\item 关于 \mintinline{cpp}|std::launch::async| ,需要多说几句。
    标准里只说异步,没说如何做到异步。
    gcc-11 的实现是简单粗暴地直接开一个新物理线程。
    我们知道乱开物理线程会闯祸。
    所以,这个特性不用也罢。
}

\begin{frame}[fragile]
    \frametitle{std::lauch::async}
    \begin{alertblock}{2 forbidden rules}
        \begin{itemize}
            \item 不可以使用 \mintinline{cpp}|std::lauch::async|
            \item 不可以使用单参数 \mintinline{cpp}|std::async|
        \end{itemize}
    \end{alertblock}
\end{frame}

\begin{frame}[fragile]
    \frametitle{Promise}
    \scriptsize
    \begin{columns}
        \begin{column}{.5\textwidth}
            \begin{exampleblock}{future from a promise}
                \begin{cppcode}
                    std::promise<int> p;
                    std::future<int> f = p.get_future();
                    std::thread t(
                        [&p]{ p.set_value(9); });
                    t.join();
                \end{cppcode}
            \end{exampleblock}
        \end{column}
        \begin{column}{.5\textwidth}
            \begin{exampleblock}{future from a packaged task}
                \begin{cppcode}
                    std::packaged_task<int()> task(
                        []{ return 7; });
                    std::future<int> f = task.get_future();
                    std::thread t(std::move(task));
                    t.join();
                \end{cppcode}
            \end{exampleblock}
        \end{column}
    \end{columns}
\end{frame}

\note[itemize]{
\item 使用异步future的正确方式是自行维护其工作线程。
    Promise更底层。
    Packaged task提供了task的抽象,略高层些。
\item Future可以视作一个只能放一个值的队列。
    \mintinline{cpp}|std::future| 是其读的一侧,
    那么 \mintinline{cpp}|std::promise| 则是其写的一侧。
}

\section{Implement a Thread Pool}

\begin{frame}
    \frametitle{Implement a Thread Pool}
    \begin{block}{thread pool, tasks and assumptions}
        \begin{itemize}
            \item 流控
                \begin{itemize}
                    \item<2-> 我们假定用户接受阻塞等待式流控。
                \end{itemize}
            \item 任务可否动态拆分
                \begin{itemize}
                    \item<3-> 我们假设任务不可动态拆分。
                \end{itemize}
            \item 任务的执行:有无优先级、有无执行序的保证
                \begin{itemize}
                    \item<4-> 我们假设任务无优先级、各自独立。
                \end{itemize}
            \item 优化目标: throughput, latency
                \begin{itemize}
                    \item<5-> 我们假定对throughput优化。
                \end{itemize}
        \end{itemize}
    \end{block}
\end{frame}

\note[itemize]{
\item 流控是工业化线程池的标配。
    没有就有爆内存的风险。
\item 任务的拆分。
    比如任务是在一个数组上执行mapping。
    性能最好的粒度一般不是每个数组元素一个任务,当然一般也不是整个数组一个任务。
    假设测试下来发现单个线程执行100个元素效率最高。
    那么肯定每100个元素组一个任务。
    如果现在线程池里有线程闲下来了,而一个忙线程正好刚开始处理一个任务。
    那么这个忙线程能不能把任务拆50个元素给那个闲线程呢?
\item 有些线程池严格保证执行序等于提交序。
    这实际隐式保证了因果关系。
    但这不是必须的。
    更好的方式是前一个任务的最后再提交一个任务,以此显式保证因果关系。
}

\begin{frame}[fragile]
    \frametitle{Implement a Thread Pool}
    \scriptsize
    \begin{exampleblock}{a naive impl.}
        \begin{cppcode}
            class ThreadPool {
                const size_t q_size_limit;
                std::mutex mtx;
                std::condition_variable cv;
                std::deque<std::function<void()>> task_q;
            }
        \end{cppcode}
    \end{exampleblock}
    \begin{columns}[t]
        \begin{column}{.5\textwidth}
            \begin{exampleblock}{ThreadPool::worker()}
                \tiny
                \begin{cppcode}
                    for(;;) {
                        function<void()> task;
                        {
                            lock_guard<mutex> g(mtx);
                            if (!task_q.empty()) {
                                swap(task, task_q.front());
                                task_q.pop_front();
                            }
                        }
                        if (task) {task();}
                        else {
                            unique_lock<mutex> g(mtx);
                            cv.wait(g);
                        }
                    }
                \end{cppcode}
            \end{exampleblock}
        \end{column}
        \begin{column}{.5\textwidth}
            \begin{exampleblock}{ThreadPool::submit()}
                \tiny
                \begin{cppcode}
                    assert(task);
                    for(;;) {
                        lock_guard<mutex> g(mtx);
                        if (task_q.size() < q_size_limit) {
                            task_q.push(std::move(task));
                            break;
                        }
                    }
                    cv.notify_one();
                \end{cppcode}
            \end{exampleblock}
        \end{column}
    \end{columns}
\end{frame}

\note[itemize]{
\item 一个实际的实现还有一些额外的细节需要考虑。
    这里就不展开了。
    现代OS对同步原语的优化已经很深入了。
    多数场景这个naive实现也足够了。
\item 这里需要提一句:用户传入的任务必须在临界区以外执行。
    否则,因为用户可能会在任务里提交任务,就死锁了。
\item 如果我们还要继续优化,我们发现对mtx的争抢是很严重的。
    核数越多,争抢越严重。
}

\begin{frame}
    \frametitle{Implement a Thread Pool}
    \begin{block}{opt ideas}
        \begin{itemize}
            \item 每个worker配一个独立的队列
            \item<2-> work stealing
                \begin{itemize}
                    \item 不要每次只偷一个
                    \item 不要所有人偷所有人
                \end{itemize}
        \end{itemize}
    \end{block}
\end{frame}

\note[itemize]{
\item 对mtx的争抢本质是对队列的争抢。
    那么很自然的,推到极端,每个worker各自一个队列,就不会有争抢了。
    然而,如果任务的耗时并不一致,那么就很容易出现“忙的忙死、闲的闲死”。
    显然,闲的应该去帮忙的干点活。
\item work stealing在实现上需要注意两个要点。
    偷意味着争抢。
    \begin{itemize}
        \item 每次偷一个。
            那么做完了就还要偷。
            这意味更多的争抢。
        \item 所有人偷所有人意味着只要有人闲下来,所有人都受影响。
    \end{itemize}
    实践中常用的做法
    \begin{itemize}
        \item 每个worker有自己的私有队列,整体一个全局队列。
            worker每次找全局队列取一个小批,自己全做完再去取。
            别人不会来偷自己私有队列的任务。
            典型是tcmalloc。
        \item 在上述逻辑的基础上将worker分组,每组一个队列。
            worker自己的私有队列就不要了。
        \item 把大家组成一个环:我只偷你的,你只偷他的,我不直接偷他的。
            那么假设有个人队列是满的,别人都是空的。
            那么他下游的那个人会分掉他一半的任务,
            下游的下游又会分掉下游的一半任务,
            依次类推,
            最后所有人都有活干,大家差不多忙。
    \end{itemize}
}

\begin{frame}[fragile,t]
    \frametitle{Implement a Thread Pool}
    \scriptsize
    \begin{exampleblock}{entangled queue}
        \tiny
        \begin{cppcode}
            class ThreadPool {
                class Slot {
                    std::mutex mtx;
                    std::function<void()> task;
                };
                std::vector<Slot> slots;
            };
        \end{cppcode}
    \end{exampleblock}
    \begin{onlyenv}<1>
        \begin{exampleblock}{worker}
            \tiny
            \begin{cppcode}
                size_t step = ...; // a random value in [0, slots.size())
                size_t cur = ...; // a random value in [0, slots.size())
                for(;;) {
                    for(size_t wakeup = K; wakeup > 0; --wakeup) {
                        Slot& cur_slot = slots[cur];
                        function<void()> task;
                        {
                            lock_guard<mutex> g(cur_slot.mtx);
                            std::swap(task, cur_slot.task);
                        }
                        if (task) {
                            task();
                            wakeup = K + 1;
                        }
                        cur = (cur + step) % slots.size();
                    }
                    this_thread::sleep_for(...);
                }
            \end{cppcode}
        \end{exampleblock}
    \end{onlyenv}
    \begin{onlyenv}<2>
        \begin{exampleblock}{submit}
            \tiny
            \begin{cppcode}
                size_t step = ...; // a random value in [0, slots.size())
                size_t cur = ...; // a random value in [0, slots.size())
                for(;;) {
                    for(size_t wakeup = K; wakeup > 0; --wakeup) {
                        Slot& cur_slot = slots[cur];
                        {
                            lock_guard<mutex> g(cur_slot.mtx);
                            if (!cur_slot.task) {
                                cur_slot.task = std::move(task);
                                return;
                            }
                        }
                        cur = (cur + step) % slots.size();
                    }
                    this_thread::sleep_for(...);
                }
            \end{cppcode}
        \end{exampleblock}
    \end{onlyenv}
\end{frame}

\note[itemize]{
\item 这个算法我没有在别处看到过。
    在阿里云某个云产品上用下来,效果很好。
    基本上那个环节4-5年里都不再是性能的瓶颈。
\item submit和worker的核心逻辑是一样的,只谈worker就好。
    每个worker都有自己的起点和步长。
    从这个起点和步长出发,整个slots可以逻辑上看作一个队列。
    也就是说,每个worker都(逻辑上)有自己的队列,但是互相纠缠在一起。
    于是,一个worker在扫自己队列的过程也自然完成了偷其他人的效果。
    同时,worker在slots上的分布可以近似认为是随机的,那么互相之间的争抢也很少。
\item 一个关键细节:
    如何保证每个任务最终都会被执行到?
    \begin{itemize}
        \item \mintinline{cpp}|slots.size()| 是一个素数。
    \end{itemize}
\item 又一个关键细节:
    K是什么?
    \begin{itemize}
        \item K是一个magic number,一个常数。
            其语义是,worker每次去检查有没有可以近似认为是对slots的充足率一次随机检测。
            现在连续K次检测都没找着活,那么我就有几分的信心相信slots的充足率一定低于几个百分点。
            实践中,比如slots 100个元素,那么低于1\%就可以当成空的。
    \end{itemize}
\item 其他还有很多实现上的细节,这里就不展开讲了。
}

\begin{frame}[plain]
    \begin{center}
        \includegraphics{pics/secret_kungfu.jpg}
    \end{center}
\end{frame}

\begin{frame}
    \frametitle{References}
    \begin{center}
            \includegraphics[height=.8\textheight]{pics/C++ Concurrency in Action.png}
    \end{center}
\end{frame}

\end{document}
