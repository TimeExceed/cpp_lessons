\documentclass[UTF8,lualatex]{ctexbeamer}
\usepackage{tabu}
\usepackage{amsmath}
\usepackage{graphicx}
\usepackage{minted}
\usepackage[normalem]{ulem}

\usetheme{Frankfurt}
\setbeamertemplate{note page}[plain]
% $BEAMER_NOTES

\title{\kaishu Modern C++ and Beyond\\
    \small
    第六课\\
    Error Handling}
    \author{陶大}
    \date{2022年9月}

\newminted{cpp}{autogobble}
\newminted{java}{autogobble}

\renewcommand{\emph}[1]{\textbf{#1}}

\newcommand{\pagequote}[2]{
  \Large
  \begin{quotation}
      #1
  \end{quotation}
  \flushright\normalsize --- {#2}
}

\begin{document}
\songti

\begin{frame}[plain]
    \titlepage
\end{frame}

\begin{frame}[plain]
    \begin{center}
        \includegraphics[height=.8\textheight]{pics/kungfu_category.jpg}
    \end{center}
\end{frame}

\begin{frame}[plain]
    \tableofcontents
\end{frame}

\section{What are errors and how to handle them?}
\subsection{What are errors?}

\begin{frame}
    \begin{block}{What are errors?}
        \begin{itemize}
            \item error: \emph{expected} conditions, must prematurely stop.
            \item panic: something \emph{unexpected} happens
        \end{itemize}
    \end{block}
\end{frame}

\note[itemize]{
\item error的语义有两重:
    对于一个特定的组件(程序、函数)
    \begin{enumerate}
        \item 组件内无法tolerant,只能终止本组件的执行(prematurely stop);
        \item 然而该状态又在预期中(expected)。
    \end{enumerate}
    意味着,其用户(或其用户的用户……)有能力处理好。
\item panic则完全处于预期之外。
    By definition, 你不知道如何正确地处理此类情况。
    因此,尽快结束整个程序(或系统)以避免错误的扩散才对。
\item 很多语言都有两类错误的区分。
    Java有Exception和Error。
    C++也有catchable/uncatchable exception。
\item 这两者的区别关键在于预期。
    而API设计的重要一环是明确预期什么。
    假如你在设计一个除法函数。
    那么除数为零究竟是一个error还是一个panic?
    取决于你如何设计API(很大程度由该除法函数的使用环境决定)。
    比如对于excel来讲,除零只能是个error,不能是panic。
}

\subsection{Error code}

\begin{frame}[fragile]
    \frametitle{Error handling: error code}
    \begin{exampleblock}{~}
        \small
        \begin{cppcode}
            int r = system("/usr/bin/ls");
            assert(r == 0);
        \end{cppcode}
    \end{exampleblock}
    \begin{block}{Error Code}
        \begin{itemize}
            \item 好处:简单
            \item 坏处
                \begin{itemize}
                    \item unintentional lost
                    \item 全局维护错误码表
                    \item 细节不丰富
                    \item 大量boilerplate code
                \end{itemize}
        \end{itemize}
    \end{block}
\end{frame}

\begin{frame}[fragile]
    \frametitle{Error handling: error code}
    \begin{exampleblock}{Unintentional lost}
        \footnotesize
        \begin{cppcode}
            [[nodiscard]] int f() {
                return 0;
            }

            int main() {
                f();
                return 0;
            }
        \end{cppcode}
        \begin{minted}[autogobble]{console}
            a.cpp: In function ‘int main()’:
            a.cpp:6:4: warning: ignoring return value of ‘int f()’, declared with attribute ‘nodiscard’ [-Wunused-result]
                6 |   f();
                  |   ~^~
        \end{minted}
    \end{exampleblock}
\end{frame}

\note[itemize]{
\item
\item 双方括号是C++11增加的语法,称为attribute specifier。
    \mintinline{text}!nodiscard!是C++17增加的attribute。
    用nodiscard修饰的函数的返回值不可以直接丢弃。
    但是,接到了不用,编译器是不管的。
}

\begin{frame}[fragile]
    \frametitle{Error handling: error code}
    \begin{block}{细节不丰富}
        \begin{itemize}
            \item 把细节存在别处,另外提供API取之
            \item 但别处不能是全局变量(不可重入)
                \begin{itemize}
                    \item 选择一:thread-local storage
                    \item 选择二:state对象
                \end{itemize}
        \end{itemize}
    \end{block}
    \begin{exampleblock}{thread-local storage, TLS}
        \footnotesize
        \begin{cppcode}
            thread_local unsigned int rage = 1;
        \end{cppcode}
    \end{exampleblock}
    \begin{exampleblock}{state对象}
        \footnotesize
        \begin{cppcode}
            typedef struct z_stream_s {
                ...
                z_const char *msg;  /* last error message, NULL if no error */
                ...
            } z_stream;
        \end{cppcode}
    \end{exampleblock}
\end{frame}

\note[itemize]{
\item TLS,它是thread local的,其他线程看不到。
    加了\mintinline{cpp}!thread_local!修饰的变量,每个用到的线程都有各自独立的一份值。
    每个线程(除非显式共享引用给其他线程)的访问都限制在本线程的那一份上,不用加锁。
    \begin{itemize}
        \item 我认为用TLS保存错误对象是不合适的。
            问题在于错误对象的生命期的管理。
            设想你写了一个库,然后你把它丢到一个线程池上去调用。
            结果每个工作线程都多出了一个错误对象。
        \item 其核心还是抽象的错配。
            TLS虽然local,但仍然是一个运行期全局的资源。
            为一个库的错误处理去占用全局的资源,就不合适。
    \end{itemize}
\item State对象没有TLS的问题。
    它的问题是每个业务(内部)状态和错误属于不同的关注切面(seperation of concern)。
    \begin{itemize}
        \item 为了保存错误,即使不会改变状态的函数也必须接受mutable对象。
        \item 错误处理的生态就比较难围绕state对象长起来。
            因为每个业务需要的state是不同的,而错误状态是嵌在里面的一行。
    \end{itemize}
}

\begin{frame}
    \frametitle{Error handling: error code}
    \begin{block}{Summary on Error Code}
        \begin{itemize}
            \item \sout{unintentional lost}
            \item 全局维护错误码表
            \item \sout{细节不丰富}
            \item 大量boilerplate code
        \end{itemize}
    \end{block}
\end{frame}

\subsection{Exceptions}

\begin{frame}[fragile]
    \frametitle{Exceptions}
    \begin{exampleblock}{~}
        \footnotesize
        \begin{cppcode}
            try {
                doSomething();
            } catch(const std::exception& ex) {
                handleError(ex);
            }
        \end{cppcode}
    \end{exampleblock}
    \begin{block}{好处}
        \begin{itemize}
            \item Object-oriented
                \begin{itemize}
                    \item 细节保存在异常对象里
                    \item 分散维护异常对象,但一般组织在同一棵类型树上
                \end{itemize}
            \item Stack unwinding
                \begin{itemize}
                    \item \emph{错误检测与错误处理分离}
                    \item 消除boilerplate code
                    \item 不可忽略
                \end{itemize}
            \item High performance
                \begin{itemize}
                    \item 在正常路径上,runtime几乎没有性能损失
                    \item 但是通常会有剧烈的生成代码膨胀,以及阻止一些有效的编译优化
                \end{itemize}
        \end{itemize}
    \end{block}
\end{frame}

\begin{frame}[fragile]
    \frametitle{Exceptions}
    \begin{exampleblock}{Pokémon exception handling (catch-all exception handling)}
        \footnotesize
        \begin{cppcode}
            try {
                doSomething();
            } catch(...) {
                // do essentially nothing but ignore
            }
        \end{cppcode}
    \end{exampleblock}
    \begin{alertblock}{Rule}
        \begin{itemize}
            \item Catching must be specific!
                Do never catch anything you don't fully understand.
        \end{itemize}
    \end{alertblock}
\end{frame}

\note[itemize]{
\item 有一些错误很难处理,一般也不需要处理,挂掉进程是更合理的选择。
    比如bad alloc。
    不要假装你能处理好。
\item C++中还有一些uncatchable exceptions,故名思义,不可以抓(至少不可以抓了不再丢出来)。
    比如\mintinline{cpp}!pthread_cancel!在C++程序里就会触发一个uncatchable exception。
}

\begin{frame}[fragile]
    \frametitle{Checked exceptions}
    \begin{exampleblock}{~}
        \footnotesize
        \begin{cppcode}
            void f() noexcept;
            void g() throw(std::exception); // until C++17
        \end{cppcode}
    \end{exampleblock}
    \begin{block}{Checked exceptions}
        Exceptions that a method may raise are part of the method's signature.
    \end{block}
\end{frame}

\note[itemize]{
\item error(非panic)既然是预期中的,那么最终总要有人来妥善处理。
    所以,error是API的一部分。
    所以,理论上,显式在函数签名上体现出来是一件好事。
\item dynamic exception specifications (\mintinline{cpp}!throw!) 这个语言特性在C++17之后从标准中移除了。
}

\begin{frame}[fragile]
    \frametitle{Criticism}
    \begin{exampleblock}{~}
        \footnotesize
        \begin{cppcode}
            double t_vth, t_vl, slew;
            gateDelays(ceff, t_vth, t_vl, slew);
            double dt = slew / (vh_ - vl_);
            double t0 = t_vth + log(1.0 - vth_) * rd_ * ceff - vth_ * dt;
            newtonRaphson(100, x_, nr_order_, driver_param_tol, evalDmpEqnsState,
                    this, fvec_, fjac_, index_, p_, scale_);
            if (debug_->check("dmp_ceff", 4))
                showVo();
        \end{cppcode}
    \end{exampleblock}
    \begin{block}{~}
        \begin{itemize}
            \item 复杂且视觉上不明显的控制流
            \item 因而,容易导致资源泄漏
        \end{itemize}
    \end{block}
    \begin{alertblock}{Automatic Resource Management}
        资源必须自动管理,禁止手动管理。
        \begin{itemize}
            \item defer in Go, try-with-statement in Java, RAII in C++/Rust
        \end{itemize}
    \end{alertblock}
\end{frame}

\note[itemize]{
\item 这段程序(取自OpenROAD/sta)会抛出异常吗?在哪里抛出?
    \begin{itemize}
        \item \mintinline{cpp}!evalDmpEqnsState! 会
        \item \mintinline{cpp}!showVo! 似乎不会
    \end{itemize}
}

\begin{frame}[fragile]
    \frametitle{Criticism}
    \begin{block}{Lots of corner cases}
        What if an exception throws from
        \begin{itemize}
            \item a ctor
                \begin{itemize}
                    \item recyclable partial-constructed objects
                \end{itemize}
            \item a dtor
                \begin{itemize}
                    \item \textcolor{red}{forbidden!}
                \end{itemize}
            \item a signal handler
                \begin{itemize}
                    \item \textcolor{red}{do NEVER write signal handlers!}
                \end{itemize}
        \end{itemize}
    \end{block}
    \begin{exampleblock}{Signal handler from OpenROAD}
        \footnotesize
        \begin{cppcode}
            static void handler(int) {
                std::cerr << "Stack trace:\n";
                std::cerr << boost::stacktrace::stacktrace();
                exit(1);
            }
        \end{cppcode}
    \end{exampleblock}
\end{frame}

\note[itemize]{
\item 析构函数不能抛异常这条很挑战RAII。
    关闭文件句柄是可能出错的。
    \begin{itemize}
        \item 所有和文件句柄关联的资源都需要在封装时给一个不抛异常的close方法。
        \item 写析构函数的时候也要很小心:能处理任何可能出错的点,而且还要能回收干净资源。
    \end{itemize}
\item 关于signal handler,要多说几句。
    signal handler非常难写:绝对不能触发内核调度。
    所以,
    \begin{itemize}
        \item 不可以做IO
        \item 不可以用锁(spinlock除外)
        \item 不可以触发异常(有些平台用signal实现异常的触发)
        \item 不可以分配内存(缺页错误会触发内核调度)
    \end{itemize}
}

\begin{frame}
    \frametitle{Criticism}
    \begin{block}{~}
        The throws clause, at least the way it's implemented in Java,
        doesn't necessarily force you to handle the exceptions,
        but if you don't handle them,
        it forces you to acknowledge precisely which exceptions might pass through.
        \emph{It requires you to either catch declared exceptions or put them in your own throws clause.}
        \\[1ex]
        To work around this requirement, people do ridiculous things.
        For example, \emph{they decorate every method with, ``throws Exception."}
        That just completely defeats the feature, and you just made the programmer write more gobbledy gunk.
        That doesn't help anybody.

        \flushright --- Anders Hejlsberg
    \end{block}
\end{frame}

\note[itemize]{
\item Anders Hejlsberg,语言设计者,代表作有C\#和TypeScript。
    Turbo Pascal编译器作者。
\item 他的这段话,其实讲了一个至今没有很好解决的问题。
    理论上,程序员应该深思熟虑,妥善处理自己收到的各种错误。
    像一个手艺人处理自己手里的材料,针对每一份材料的特性进行处理,尽量包容材料的缺陷。
    但实际上,很多程序员是工业流水线上的一环。
    不被测试发现问题就好;如果被发现了,能甩锅也行。
    \begin{itemize}
        \item 原则上,错误不应该跨过模块。
            比如IO错误,一个IO库当然可以丢出来。
            但是其用户应该要么处理掉,要么包装成自己的错误(同时提供更多诊断信息),
            而不应该直接把底层的IO错误再往上丢,导致自己的用户依赖自己的依赖。
    \end{itemize}
}

\subsection{Nullptr as absence}

\begin{frame}
    \includegraphics[width=\textwidth,height=\textheight]{pics/nullptr.jpeg}
\end{frame}

\note[itemize]{
\item 这是Hoare老先生在2009给的一个talk的标题。
    Hoare在1965(ALGOL W)引入的空指针,``simply because it was so easy to implement''.
}

\begin{frame}[fragile]
    \frametitle{Nullptr as absence}
    \begin{exampleblock}{Nullptr as absence}
        \footnotesize
        \begin{cppcode}
            void* p = malloc(...);
            char* s = strdup(...);
            // On error, `p`/`s` will be nullptr.
        \end{cppcode}
    \end{exampleblock}
\end{frame}

\note[itemize]{
\item NAA有生命期的问题,这是C++特有的问题:
    比如在\mintinline{text}!strdup!例子里,返回的对象只能分配在堆上或是全局对象。
    这会带来不必要的性能问题。
    另外,用户也要自己负责返回对象的销毁。
\item 除此之外,(即使在有GC语言里),NAA也有问题。
}

\begin{frame}[fragile]
    \frametitle{Nullptr as absence}
    \begin{columns}[t]
        \footnotesize
        \begin{column}{.5\textwidth}
            \begin{alertblock}{错误的写法(in Java)}
                \begin{javacode}
                    class A {
                        B g() {...}
                    }
                    class B {
                        C h() {...}
                    }

                    static C f(A a) {
                        return a.g().h();
                    }
                \end{javacode}
            \end{alertblock}
        \end{column}
        \begin{column}{.5\textwidth}
            \begin{exampleblock}{正确的写法(before Java-8)}
                \begin{javacode}
                    static C f(A a) {
                        if (a == null) {
                            return null;
                        }
                        B b = a.g();
                        if (b == null) {
                            return null;
                        }
                        return b.h();
                    }
                \end{javacode}
            \end{exampleblock}
        \end{column}
    \end{columns}
\end{frame}

\note[itemize]{
\item 大量boilerplate code。
    所以,工程师都不愿意处理nullptr。
    导致生产中大量的nullptr exception (NPE)。
}

\subsection{std::optional and std::expected}

\begin{frame}[fragile]
    \frametitle{std::optional}
    \begin{exampleblock}{~}
        \footnotesize
        \begin{cppcode}
            optional<B> A::g();
            optional<C> B::h();
        \end{cppcode}
    \end{exampleblock}
    \begin{columns}[t]
        \footnotesize
        \begin{column}{.45\textwidth}
            \begin{exampleblock}{in C++17}
                \begin{cppcode}
                    optional<C> f(optional<A> a)
                    {
                        if (a) {
                            return nullopt;
                        }
                        B b = a->g();
                        if (b) {
                            return nullopt;
                        }
                        return b->h();
                    }
                \end{cppcode}
            \end{exampleblock}
        \end{column}
        \begin{column}{.55\textwidth}
            \begin{exampleblock}{in C++23}
                \begin{cppcode}
                    optional<C> f(optional<A> a)
                    {
                        return a
                            .and_then([](A&& a) {a.g()})
                            .and_then([](B&& b) {b.h()});
                    }
                \end{cppcode}
            \end{exampleblock}
        \end{column}
    \end{columns}
\end{frame}

\begin{frame}[fragile]
    \frametitle{std::expected}
    \begin{exampleblock}{error handling in C++23}
        \footnotesize
        \begin{cppcode}
            std::expected<int, std::string> f() noexcept {
                if (...) {
                    return std::unexpected("something bad happened");
                }
                // everything is ok
                return 42;
            }

            void g() noexcept {
                auto r = f();
                if (!r) {
                    auto& err = r.error();
                    // deal with `err`
                } else {
                    auto& val = r.value();
                    // deal with `val`
                }
            }
        \end{cppcode}
    \end{exampleblock}
\end{frame}

\note[itemize]{
\item std::expected类似Rust的Result,但成熟度远远不足
    (比如没有\mintinline{cpp}!and_then!)
    所以,看来2026年之前都用不上。
\item 吐个槽:C++在各方面模仿Rust,
    无论optional/expected之于Option/Result、ranges之于Iterator、coroutine、module、formatting。
    还没有safety保证。
    那么,为什么还用C++呢?
}

\begin{frame}[fragile]
    \frametitle{A suggestion}
    \begin{exampleblock}{~}
        \footnotesize
        \begin{cppcode}
            #define TRY(expr) \
                do { \
                    const auto& res36280 = (expr); \
                    if (res36280) {\
                        return res36280; \
                    } \
                } while(false)

            [[nodiscard]] std::optional<Error> f(int& result) noexcept;

            [[nodiscard]] std::optional<Error> g() noexcept {
                int result = 0;
                TRY(f(result));
            }
        \end{cppcode}
    \end{exampleblock}
\end{frame}

\note[itemize]{
\item 我自己的实践,这样做虽然不完美,但也足够方便。
    \\
    \url{https://github.com/TimeExceed/aliyun-tablestore-cpp-sdk}
}

\section{Monad, a pattern on seperation  of concerns}

\begin{frame}
    \frametitle{Monad}
    \begin{block}{Famous monads}
        \begin{itemize}
            \item Option/Maybe monad: represent an optional value
            \item Result/Either monad: represent a successful operation or an error
            \item IO monad: represent side-effects
            \item Future/Task monad: represent lazy/async computation
            \item List monad: represent computations on iterators/streams
            \item $\cdots$
        \end{itemize}
    \end{block}
    \begin{center}
        \Large
        What's a monad?
    \end{center}
\end{frame}

\begin{frame}
    \frametitle{Monad}
    \pagequote
    {A monad is a monoid in the category of endofunctors, what's the problem?}
    {Philip Wadler}
\end{frame}

\note[itemize]{
\item 开个玩笑。
    我这里不准备把大家培训成数学家,不准备讲幺半群、自函子和自函子上的范畴。
    我准备给大家从程序员视角讲一讲。
}

\begin{frame}[fragile]
    \frametitle{Monad, from a programmer's perspective}
    \begin{columns}[t]
        \footnotesize
        \begin{column}{.5\textwidth}
            \begin{exampleblock}{a Maybe monad}
                \begin{cppcode}
                    auto k = make_optional(1)
                        .and_then([](int x) {
                            return optional<int>(x + 1);
                        })
                        .transform([](int x) {
                            return x + 2;
                        });
                \end{cppcode}
            \end{exampleblock}
        \end{column}
        \begin{column}{.5\textwidth}
            \begin{exampleblock}{a Future monad (in Rust, 伪代码)}
                \begin{minted}[autogobble]{rust}
                    open("/path/to/file")
                        .then(|f| f.read_all())
                        .then(|c| stdout.write(&c));
                \end{minted}
            \end{exampleblock}
        \end{column}
    \end{columns}
\end{frame}

\note[itemize]{
\item C++的std::future没有实现成monad。
    也许类似optional,以后会做。
    所以这里用rust的future作为示例。
\item 从这两个例子里,我们可以看到monad的三个要素:
    \begin{itemize}
        \item 一个能往里塞类型的框框。
            框框,有些文献称为容器(container。不要和stl容器混淆),有些文献称为context。
            这个框框是有特定语义的,所以叫context。
            \begin{itemize}
                \item optional的语义就是,顾名思义,optional
                \item future的语义就是一个尚未发生的计算。
                    Future例子里的`k',并不是future of 4,而是
                    future of a computation, whose result is 4.
            \end{itemize}
            往里塞的东西,不是对象,是\emph{类型}。
        \item 一个把普通值装进monad value的函数(return or pure function)。
        \item 从普通值映射到monad value的函数(bind or flat-map function)。
        \item 一般各种monad的实现也会提供接受把普通值到普通值的函数扩展成bind function的接口。
    \end{itemize}
\item Monad最重要的是那个框框。
    它提供的核心价值是通过分离关注面(seperation of concerns)来消除业务逻辑之外的boilerplate code。
}

\begin{frame}[fragile]
    \frametitle{Future monad}
    \begin{columns}[t]
        \footnotesize
        \begin{column}{.5\textwidth}
            \begin{exampleblock}{by callbacks (in Rust, 伪代码)}
                \scriptsize
                \begin{minted}[autogobble]{rust}
                    executor.open("/path/to/file", |f| {
                        executor.read_all(f, |c| {
                            executor.write(stdout, &c);
                        })
                    });
                \end{minted}
            \end{exampleblock}
        \end{column}
        \begin{column}{.5\textwidth}
            \begin{exampleblock}{by Future monads (in Rust, 伪代码)}
                \begin{minted}[autogobble]{rust}
                    let fut = open("/path/to/file")
                        .then(|f| f.read_all())
                        .then(|c| stdout.write(&c));
                    executor.run(fut);
                \end{minted}
            \end{exampleblock}
        \end{column}
    \end{columns}
\end{frame}

\note[itemize]{
\item 我们知道,在线程池(executor)上做blocked IO是大忌。
    为了充分利用CPU,早先的主流用回调,然后是Future monad,现在用async/await。
    async/await以前的几课里讲过,需要改语言。
\item 回调有两个问题:一曰丑、一曰难。
    丑是指缩进一层套一层。
    AJAX时代的web前端开发应该有体会:一屏代码半屏缩进。
    难有两个层面:
    \begin{enumerate}
        \item 第一个层面,一个连贯的业务逻辑被IO打断,
            需要用户能把断点之前的状态传递到断点之后。
            如何高效做到状态管理,这就考验人的思考力和经验。
        \item 第二个层面,IO框架的运行期管理也是个难题。
            如例子中所示,IO框架对象(\mintinline{text}!executor!)(几乎)被所有回调所使用。
            所以executor的生命无法静态确定。
            如果使用shared\_ptr(或类似的reference counting GC技术)
            使用的时候也需要留意避免executor去依赖回调造成GC不掉。
    \end{enumerate}
\item ~
    \begin{itemize}
        \item 显然Future monad解决了丑的问题。
        \item 我们注意到,\mintinline{text}!fut!实际上里面只有业务逻辑,无关任何框架。
            通过分离业务逻辑和如何驱动业务逻辑,它也解决了难的第二个层面问题。
        \item 难的第一个层面问题,monad并没有直接解决。
            但是武器库中有一种特殊的monad,称为state monad。
            自然理解起来更复杂一丢丢。
            这里就不展开了。
    \end{itemize}
}

\begin{frame}[fragile]
    \frametitle{What's a Monad?}
    \begin{columns}
        \begin{column}{.5\textwidth}
            \begin{exampleblock}{a Maybe monad}
                \footnotesize
                \begin{cppcode}
                    auto k = make_optional(1)
                        .and_then([](int x) {
                            return optional<int>(x+1);
                        })
                        .transform([](int x) {
                            return x + 2;
                        });
                \end{cppcode}
            \end{exampleblock}
        \end{column}
        \begin{column}{.5\textwidth}
            \begin{block}{Monad, from a programmer's perspective}
                A pattern to remove boilerplate code about \emph{interaction between framework and business}.
                \begin{itemize}
                    \item framework: 一个纵贯业务逻辑的功能点。
                    \item 一个完整的业务逻辑可以拆分成多段计算。
                        每段计算结束之后和framework有所交互。
                \end{itemize}
            \end{block}
        \end{column}
    \end{columns}
\end{frame}

\note[itemize]{
\item 在optional,功能点就是计算的短路能力。
    在future,功能点就是异步计算能力。
}

\begin{frame}
    \frametitle{Revisit}
    \tableofcontents
\end{frame}

\begin{frame}[plain]
    \begin{center}
        \includegraphics{pics/secret_kungfu.jpg}
    \end{center}
\end{frame}

\begin{frame}
    \frametitle{References}
    \begin{itemize}
        \item \url{https://en.wikipedia.org/wiki/Monad_(functional_programming)}
        \item A Monad is just a Monoid… by Michele Stieven \url{https://michelestieven.medium.com/a-monad-is-just-a-monoid-a02bd2524f66}
    \end{itemize}
\end{frame}

\end{document}


