\documentclass[UTF8,lualatex]{ctexbeamer}
\usepackage{tabu}
\usepackage{amsmath}
\usepackage{graphicx}
\usepackage{minted}

\usetheme{Frankfurt}
\setbeamertemplate{note page}[plain]
% $BEAMER_NOTES

\title{\kaishu Modern C++ and Beyond\\
    第一课:编译、链接与构建系统}
    \author{陶大}
    \date{2022年7月}

\newminted{cpp}{autogobble}
\newminted{bash}{autogobble}
\newminted{console}{autogobble}

\begin{document}
\songti

\begin{frame}[plain]
    \titlepage
\end{frame}

\begin{frame}[plain]
    \begin{center}
        \includegraphics[height=.8\textheight]{pics/kungfu_category.jpg}
    \end{center}
\end{frame}

\begin{frame}[plain]
    \tableofcontents
\end{frame}

\begin{frame}
    \frametitle{}
    \begin{block}{若干疑惑}
        \begin{itemize}
            \item 编译个程序为什么那么复杂?
                Make, SCons, Bazel, CMake都是什么鬼?
            \item 当我运行一个程序,发生了什么?
            \item 为什么头文件里实现函数,需要加inline?
            \item \mintinline{C++}{extern "C"} 是做什么的?
                被它修饰的函数可以调用C++库吗?
            \item 动态库和静态库什么区别?
                为什么用静态库跑得好好的程序,换成动态库就崩了?
        \end{itemize}
    \end{block}
\end{frame}

\section{构建系统}

\begin{frame}[fragile=singleslide]
    \frametitle{C/C++的编译:简图}
    \begin{center}
        \includegraphics[width=\textwidth]{pics/build-brief.png}
    \end{center}
    \begin{block}{a shell script}
        \begin{bashcode}
            #!/bin/bash
            gcc -c -o a.o a.c
            g++ -c -o b.o b.cpp
            g++ -o exe a.o b.o
        \end{bashcode}
    \end{block}
\end{frame}

\note[itemize]{
\item 现代语言,比如python和java,并非没有链接。
    链接的本质是模块化。
}

\begin{frame}
    \frametitle{C/C++的编译:事情开始起变化}
    \begin{columns}
        \begin{column}{.48\textwidth}
            \includegraphics[height=0.8\textheight]{pics/build-complex.png}
        \end{column}
        \begin{column}{.5\textwidth}
            \begin{itemize}
                \item 构建图庞大、复杂
                \item 增量构建
                \item 并行构建
                \item IDE友好
            \end{itemize}
        \end{column}
    \end{columns}
\end{frame}

\note[itemize]{
\footnotesize
\item 我当年刚去阿里云的时候,优化了一把构建,从几个小时优化到几分钟。
    一举成名。
    此事影响深远,譬如daily build、TDD,从此皆有可能。
    我收获的经验是:
    \begin{enumerate}
        \item 软件工程的实践非常依赖基础设施。
            基础设施不到位,优秀的软件工程不过是空中楼阁。
        \item 然而管理基础设施的改进却是难题。
            一者不知道基础设施有问题,二者知道有问题但如之奈何,三者基础设施必定持续腐化。
    \end{enumerate}
    “诸葛在时,亦不觉异;自公殁后,不见其比。"(房玄龄《晋书》)
    犹基础设施之谓也。
\item 所谓构建图,抽象来讲就是一张DAG,提供增量构建和并行构建的能力。
    其实是一张hypergraph,有resources(方)和processors(圆)两种节点。
\item 增量构建:一个资源只有自己直接或间接依赖的(方或圆)节点发生变化了才被构建。
    比如如果修改了图中a.c,那么只有a.o和exe1需要重新构建。
    而如果修改了图中b.h,那么几乎全图都要重新刷一遍。
\item 由codegen生成的代码的场景还挺丰富。
    包括但不限于:
    \begin{itemize}
    \item 跨平台
    \item protocol/schema codegen
    \item 版本号
        \begin{itemize}
            \item ORD把版本号写进头文件的做法是有问题的。问题是什么?
        \end{itemize}
    \end{itemize}
}

\begin{frame}
    \frametitle{构建系统简史}
    \begin{block}{~}
        \begin{enumerate}
            \item 初代目
                \begin{itemize}
                    \item make: 构建图
                \end{itemize}
            \item 二代目:寒武纪物种大爆发
                \begin{itemize}
                    \item scons: 易定制
                    \item ant: IDE-friendly DSL
                    \item maven: package management
                \end{itemize}
            \item 三代目:前后端分离
                \begin{itemize}
                    \item cmake: 专心画图
                    \item ninja: 专心构建
                \end{itemize}
        \end{enumerate}
    \end{block}
\end{frame}

\note[itemize]{
\item cmake: 专心画好构建图。
    为软件的模块化开发提供了支持。
    也有插件提供包管理,但被C/C++拖累,不太成功。
    但语法老派,一股TCL味。
\item ninja: 专心构建,快。语法简单,但方便parsing。
    \begin{itemize}
        \item IDE才不关心怎么编译一个.o出来。
            IDE关心,你这个.c都include了哪些.h,都在哪里可以找到。
            而cmake生成的ninja file恰好包含了这些信息。
    \end{itemize}
\item ORD虽然用到了cmake,但还是以makefile为后端。
    并且似乎用了一些makefile上的黑魔法,不太容易把ninja用起来。
    不太容易的意思是,我搞了5分钟,没搞定。
}

\begin{frame}[fragile]
    \frametitle{fassert:一个例子}
    \begin{block}{in fassert/src}
        \small
        \begin{minted}[autogobble]{cmake}
            add_library(fassert STATIC fassert.cpp)
            target_link_libraries(fassert PUBLIC prettyprint)
            target_include_directories(fassert PUBLIC .)
        \end{minted}
    \end{block}
    \begin{block}{in user-side}
        \small
        \begin{minted}[autogobble]{cmake}
            add_subdirectory(fancy_assert/src)
            target_link_libraries(user_lib fassert)
        \end{minted}
    \end{block}
    \begin{block}{in user-side .cpp/.hpp}
        \small
        \begin{cppcode}
            #include "fassert.hpp"
        \end{cppcode}
    \end{block}
\end{frame}

\note[itemize]{
\item 每个库只需要说明自己直接依赖了什么。
    不需要处理include path,也不需要处理.a ordering。
}

\section{符号}

\subsection{什么产生符号、产生什么符号}

\begin{frame}
    \frametitle{符号}
    \begin{block}{一个例子}
        \includegraphics[width=\textwidth]{pics/gpl_unittest.png}
    \end{block}
\end{frame}

\note[itemize]{
\item 不需要执行程序就能知道程序的版本。
    这点在调查coredump特别有用。
}

\begin{frame}[fragile,t]
    \frametitle{符号}
    \scriptsize
    \begin{exampleblock}{a.c}
        \begin{cppcode}
            #define AAA 100;
            int main() {
                return 0;
            }
        \end{cppcode}
    \end{exampleblock}
    \begin{block}{~}
        \begin{consolecode}
            $ nm exe | grep AAA
        \end{consolecode}
    \end{block}
\end{frame}

\note[itemize]{
\item 什么都没有。
\item 顺带一提,这个时候AAA也没有类型。
}

\begin{frame}[fragile,t]
    \frametitle{符号}
    \scriptsize
    \begin{exampleblock}{a.c}
        \begin{cppcode}
            #define AAA 100;
            int main() {
                int a = AAA;
                return 0;
            }
        \end{cppcode}
    \end{exampleblock}
    \begin{block}{~}
        \begin{consolecode}
            $ nm exe | grep AAA
        \end{consolecode}
    \end{block}
\end{frame}

\note[itemize]{
\item 还是什么都没有。
}

\begin{frame}[fragile,t]
    \frametitle{符号}
    \scriptsize
    \begin{exampleblock}{a.c}
        \begin{cppcode}
            const int AAA = 100;
            int main() {
                return 0;
            }
        \end{cppcode}
    \end{exampleblock}
    \begin{block}{~}
        \begin{consolecode}
            $ nm exe | grep AAA
            0000000000002004 r AAA
        \end{consolecode}
    \end{block}
\end{frame}

\note[itemize]{
\item 如果没开编译优化,就有了。
    开了编译优化就没了。
\item 即使加上了赋值,也还是这样。
}

\begin{frame}[fragile,t]
    \frametitle{符号}
    \scriptsize
    \begin{exampleblock}{a.c}
        \begin{cppcode}
            #include <stdio.h>
            const int AAA = 100;
            int main() {
                const int* a = &AAA;
                printf("%p\n", a);
                return 0;
            }
        \end{cppcode}
    \end{exampleblock}
    \begin{block}{~}
        \begin{consolecode}
            $ nm exe | grep AAA
            0000000000002004 r AAA
        \end{consolecode}
    \end{block}
\end{frame}

\note[itemize]{
\item 取地址,自然需要有地址,所以会在.rodata段给安排上。
}

\begin{frame}[fragile,t]
    \frametitle{符号}
    \scriptsize
    \begin{exampleblock}{a.c}
        \begin{cppcode}
            #include <stdio.h>
            int AAA = 100;
            int main() {
                AAA++;
                int a = AAA;
                printf("%p\n", &a);
                return 0;
            }
        \end{cppcode}
    \end{exampleblock}
    \begin{block}{~}
        \begin{consolecode}
            $ nm exe | grep AAA
            0000000000004010 D AAA
        \end{consolecode}
    \end{block}
\end{frame}

\note[itemize]{
\item 要能修改,自然内存里需要占个地儿。所以也会在.data段给安排上。
}

\begin{frame}[fragile,t]
    \frametitle{符号}
    \scriptsize
    \begin{columns}[t]
        \begin{column}{.3\textwidth}
            \begin{exampleblock}{a.c}
                \begin{cppcode}
                    #include <stdio.h>
                    #include "b.h"
                    int main() {
                        int a = AAA;
                        printf("%p\n", &a);
                        return 0;
                    }
                \end{cppcode}
            \end{exampleblock}
        \end{column}
        \begin{column}{.33\textwidth}
            \begin{exampleblock}{b.h}
                \begin{cppcode}
                    #pragma once
                    extern int AAA;
                \end{cppcode}
            \end{exampleblock}
        \end{column}
        \begin{column}{.33\textwidth}
            \begin{exampleblock}{b.c}
                \begin{cppcode}
                    #include "b.h"
                    int AAA = 100;
                \end{cppcode}
            \end{exampleblock}
        \end{column}
    \end{columns}
    \begin{block}{~}
        \begin{consolecode}
            $ nm exe | grep AAA
            0000000000004010 D AAA
            $ nm a.o | grep AAA
                             U AAA
            $ nm b.o | grep AAA
            0000000000000000 D AAA
        \end{consolecode}
    \end{block}
\end{frame}

\note[itemize]{
\item 跨了两个编译单元。
    编译a.c的时候不知道AAA究竟是啥,所以编译器只能给在.data段给安排上。
\item 在a.o里AAA的标记是U (undefined)。
    意思是我这里用到了这个,但我不知道是啥。
\item 在b.o这里AAA的标记是D (在.data段)。
\item 所以我们知道了,符号是一个名字。
    在磁盘上的时候,它代表某个内容;
    在内存里的时候,它代表一个加载时候确定的地址。
}

\begin{frame}[fragile,t]
    \frametitle{符号}
    \scriptsize
    \begin{columns}[t]
        \begin{column}{.3\textwidth}
            \begin{exampleblock}{a.c}
                \begin{cppcode}
                    #include <stdio.h>
                    #include "b.h"
                    int main() {
                        int a = AAA;
                        printf("%p\n", &a);
                        return 0;
                    }
                \end{cppcode}
            \end{exampleblock}
        \end{column}
        \begin{column}{.33\textwidth}
            \begin{exampleblock}{b.h}
                \begin{cppcode}
                    #pragma once
                    extern int AAA;
                \end{cppcode}
            \end{exampleblock}
        \end{column}
        \begin{column}{.33\textwidth}
            \begin{exampleblock}{b.c}
                \begin{cppcode}
                    #include "b.h"
                \end{cppcode}
            \end{exampleblock}
        \end{column}
    \end{columns}
    \begin{onlyenv}<2>
        \begin{block}{~}
            \begin{consolecode}
                /usr/bin/ld: a.o: in function `main':
                a.cpp:(.text.startup+0x24): undefined reference to `AAA'
                collect2: error: ld returned 1 exit status
            \end{consolecode}
        \end{block}
    \end{onlyenv}
\end{frame}

\begin{frame}[fragile,t]
    \frametitle{符号}
    \scriptsize
    \begin{columns}[t]
        \begin{column}{.3\textwidth}
            \begin{exampleblock}{a.c}
                \begin{cppcode}
                    #include "b.h"
                    int main() {
                        pp();
                        return 0;
                    }
                \end{cppcode}
            \end{exampleblock}
        \end{column}
        \begin{column}{.33\textwidth}
            \begin{exampleblock}{b.h}
                \begin{cppcode}
                    #pragma once
                    void pp();
                \end{cppcode}
            \end{exampleblock}
        \end{column}
        \begin{column}{.33\textwidth}
            \begin{exampleblock}{b.c}
                \begin{cppcode}
                    #include "b.h"
                    #include <cstdio>
                    void pp() {
                        printf("Hello\n");
                    }
                \end{cppcode}
            \end{exampleblock}
        \end{column}
    \end{columns}
    \begin{block}{~}
        \begin{consolecode}
            $ nm a.o
            0000000000000000 T main
                             U pp()
            $ nm b.o
            0000000000000000 r .LC0
                             U puts
            0000000000000000 T pp()
            $ nm exe
            ...
            0000000000001170 T pp()
        \end{consolecode}
    \end{block}
\end{frame}

\note[itemize]{
\item a.o里面是U,b.o里面是T (在.text段)。
    链接之后是T。
    合乎预期。
}

\begin{frame}[fragile,t]
    \frametitle{符号}
    \scriptsize
    \begin{columns}[t]
        \begin{column}{.3\textwidth}
            \begin{exampleblock}{a.c}
                \begin{cppcode}
                    #include "b.h"
                    int main() {
                        pp();
                        return 0;
                    }
                \end{cppcode}
            \end{exampleblock}
        \end{column}
        \begin{column}{.33\textwidth}
            \begin{exampleblock}{b.h}
                \begin{cppcode}
                    #pragma once
                    #include <cstdio>
                    void pp() {
                        printf("Hello\n");
                    }
                \end{cppcode}
            \end{exampleblock}
        \end{column}
        \begin{column}{.33\textwidth}
            \begin{exampleblock}{b.c}
                \begin{cppcode}
                    #include "b.h"
                    void qq() {
                        pp();
                    }
                \end{cppcode}
            \end{exampleblock}
        \end{column}
    \end{columns}
    \begin{block}{~}
        \begin{consolecode}
            $ gcc -o exe a.o b.o
            /usr/bin/ld: b.o: in function `pp()':
            b.cpp:(.text+0x0): multiple definition of `pp()'; a.o:a.cpp:(.text+0x0): first defined here
            collect2: error: ld returned 1 exit status
            $ nm a.o
            ...
            0000000000001170 T pp()
            $ nm b.o
            ...
            0000000000000000 T pp()
        \end{consolecode}
    \end{block}
\end{frame}

\begin{frame}[fragile,t]
    \frametitle{符号}
    \scriptsize
    \begin{columns}[t]
        \begin{column}{.3\textwidth}
            \begin{exampleblock}{a.c}
                \begin{cppcode}
                    #include "b.h"
                    int main() {
                        pp();
                        return 0;
                    }
                \end{cppcode}
            \end{exampleblock}
        \end{column}
        \begin{column}{.33\textwidth}
            \begin{exampleblock}{b.h}
                \begin{cppcode}
                    #pragma once
                    #include <cstdio>
                    inline void pp() {
                        printf("Hello\n");
                    }
                \end{cppcode}
            \end{exampleblock}
        \end{column}
        \begin{column}{.33\textwidth}
            \begin{exampleblock}{b.c}
                \begin{cppcode}
                    #include "b.h"
                    void qq() {
                        pp();
                    }
                \end{cppcode}
            \end{exampleblock}
        \end{column}
    \end{columns}
    \begin{block}{~}
        \begin{consolecode}
            $ nm a.o
            ...
            0000000000000000 W pp()
            $ nm b.o
            ...
            0000000000000000 W pp()
        \end{consolecode}
    \end{block}
\end{frame}

\note[itemize]{
\item 在C/C++标准里,inline关键字只是给编译器是否内联的一个hint。
    编译器可以内联也可以不内联(现代编译器倾向于忽略)。
    所以inline关键字实际的作用是产生W符号(weak符号)。
\item 链接器会将同名的弱符号只保留一份。
    至于保留哪一份,随缘吧。
}

\begin{frame}[fragile]
    \frametitle{符号}
    \scriptsize
    \begin{exampleblock}{各种符号规则}
        \begin{cppcode}
            class A {
            public:
                const int a = 100; // 弱符号
                void b() {} // 弱符号
                void c();
                void d();
            };
            void A::c() {} // 强符号
            inline void A::d() {} // 弱符号
        \end{cppcode}
    \end{exampleblock}
\end{frame}

\note[itemize]{
\item 毫无一致性可言。
}

\subsection{Name Mangling}

\begin{frame}[fragile]
    \frametitle{Name Mangling}
    \scriptsize
    \begin{block}{~}
        \begin{cppcode}
            #pragma once
            #include <cstdio>
            inline void pp() {
                printf("Hello\n");
            }
        \end{cppcode}
        \begin{consolecode}
            $ nm b.o
                             U puts
            0000000000000000 W _Z2ppv
            0000000000000000 T _Z2qqv
        \end{consolecode}
    \end{block}
\end{frame}

\note[itemize]{
\item C++为什么要做name mangling?
    C++有函数重载、模板。C++在产生符号的时候需要编码更多信息才可以避免重名。
\item C-C++ co-operation怎么办?
}

\begin{frame}[fragile]
    \frametitle{Name Mangling}
    \scriptsize
    \begin{block}{~}
        \begin{cppcode}
            #pragma once
            #include <cstdio>
            extern "C" inline void pp() {
                printf("Hello\n");
            }
        \end{cppcode}
        \begin{consolecode}
            $ nm b.o
            0000000000000000 W pp
                             U puts
            0000000000000000 T _Z2qqv
        \end{consolecode}
    \end{block}
\end{frame}

\note[itemize]{
\item \mintinline{cpp}{extern "C"}理论上除了名字,还包括calling convention的变化。
    calling convention的意思是,
    \begin{itemize}
    \item 参数压栈的时候从左往右还是从右往左
    \item 寄存器传参的时候用哪几个寄存器
    \item 返回值放哪里
    \item 返回到哪里去
    \end{itemize}
    但一般编译器不会自己给自己找麻烦,gcc/g++不会用不同的calling convention,
    clang也不会。
}

\section{链接与装载}

\subsection{ELF}

\begin{frame}
    \frametitle{各种文件的分类}
    \begin{center}
        \includegraphics{pics/build-files.png}
    \end{center}
\end{frame}

\note[itemize]{
\item .a和.so虽然都是库,但格式其实差别很大。
    .a基本上是一堆.o打包再加上索引。
    但是我也见过把.a解压之后的.o缺符号的。
    原因不清楚。
}

\begin{frame}[fragile]
    \frametitle{ELF文件}
    \begin{block}{~}
        \scriptsize
        \begin{cppcode}
            #include <stdio.h>
            const char HELLO[] = "Hello World!";
            void print() {printf("%s\n", HELLO);}
            int main() {print();return 0;}
        \end{cppcode}
    \end{block}
    \begin{block}{~}
        \scriptsize
        \begin{bashcode}
            $ objdump --headers exe
            Idx Name    Size      VMA               LMA               File off  Algn
            11 .init    0000001b  0000000000001000  0000000000001000  00001000  2**2
                        CONTENTS, ALLOC, LOAD, READONLY, CODE
            15 .text    0000011c  0000000000001060  0000000000001060  00001060  2**4
                        CONTENTS, ALLOC, LOAD, READONLY, CODE
            16 .init_array   00000010  0000000000003d98  0000000000003d98  00002d98  2**3
                        CONTENTS, ALLOC, LOAD, DATA
            17 .rodata  00000015  0000000000002000  0000000000002000  00002000  2**3
                        CONTENTS, ALLOC, LOAD, READONLY, DATA
            24 .data    00000010  0000000000004000  0000000000004000  00003000  2**3
                        CONTENTS, ALLOC, LOAD, DATA
            25 .bss     00000008  0000000000004010  0000000000004010  00003010  2**0
                        ALLOC
            ...
        \end{bashcode}
    \end{block}
\end{frame}

\note[itemize]{
\item ELF的每个段(segment)在加载的时候会被加载到连续的内存(section)里。
    OS的内存一页(page)一般是4KB。
    所以section一般是4KB的整数倍。
    为了提高内存利用率,几个segments可能被塞到同一个section里。
    现代CPU可以给内存页打上不同的标签(可执行、可读、可写),所以只有flag相同的segments可以塞进同一个section。
\item 常见各段的含义
    \begin{itemize}
        \item .init\qquad 程序初始化函数
        \item .text\qquad 代码
        \item .rodata\qquad 只读全局数据
        \item .data/.bss\qquad 全局数据
    \end{itemize}
}

\begin{frame}[fragile]
    \frametitle{一个程序是怎么执行的?}
    \scriptsize
        \begin{block}{~}
            \begin{bashcode}
                $ objdump --disassemble exe
            \end{bashcode}
        \end{block}
        \begin{block}{~}
            \begin{minted}[autogobble]{text}
                Disassembly of section .init:

                0000000000001000 <_init>:
                endbr64
                sub    $0x8,%rsp
                mov    0x2fd9(%rip),%rax # 3fe8 <__gmon_start__@Base>
                test   %rax,%rax
                je     1016 <_init+0x16>
                call   *%rax
                add    $0x8,%rsp
                ret
            \end{minted}
        \end{block}
\end{frame}

\note[itemize]{
\item OS加载了exe之后会调用.init段的\_init函数。
    现在这个函数不做任何事情。
}

\begin{frame}[fragile]
    \frametitle{一个程序是怎么执行的?}
    \scriptsize
    \begin{block}{~}
        \begin{minted}[autogobble]{text}
            Disassembly of section .text:

            0000000000001060 <_start>:
            endbr64
            xor    %ebp,%ebp
            mov    %rdx,%r9
            pop    %rsi
            mov    %rsp,%rdx
            and    $0xfffffffffffffff0,%rsp
            push   %rax
            push   %rsp
            xor    %r8d,%r8d
            xor    %ecx,%ecx
            lea    0xe4(%rip),%rdi # 1163 <main>
            call   *0x2f53(%rip)   # 3fd8 <__libc_start_main@GLIBC_2.34>
            hlt
            cs nopw 0x0(%rax,%rax,1)
        \end{minted}
    \end{block}
\end{frame}

\note[itemize]{
\item OS然后会去找.text段的\_start函数。
\item \_start函数还是gcc生成的。
    它会把main函数作为参数调用libc-start-main函数。
    这么处理的原因是,根据C语言标准,main函数在执行之前要安排好一些事情:
    \begin{itemize}
    \item 命令行参数
    \item 环境变量
    \item C++的话还有异常捕获
    \end{itemize}
}

\begin{frame}[fragile]
    \frametitle{一个程序是怎么执行的?}
    \scriptsize
    \begin{columns}[t]
        \begin{column}{.45\textwidth}
            \begin{block}{~}
            \begin{cppcode}
                #include <stdio.h>

                const char HELLO[]
                    = "Hello World!";

                void print() {
                    printf("%s\n", HELLO);
                }

                int main() {
                    print();
                    return 0;
                }
            \end{cppcode}
            \end{block}
        \end{column}
        \begin{column}{.5\textwidth}
            \begin{minted}[autogobble]{text}
                Disassembly of section .text:

                0000000000001149 <print>:
                endbr64
                push   %rbp
                mov    %rsp,%rbp
                lea    0xeb0(%rip),%rax # 2008 <HELLO>
                mov    %rax,%rdi
                call   1050 <puts@plt>
                nop
                pop    %rbp
                ret

                0000000000001163 <main>:
                endbr64
                push   %rbp
                mov    %rsp,%rbp
                mov    $0x0,%eax
                call   1149 <print>
                mov    $0x0,%eax
                pop    %rbp
                ret
            \end{minted}
        \end{column}
    \end{columns}
\end{frame}

\note[itemize]{
\item 从这里开始才是我们在代码里能看到的main函数、print函数和HELLO常量。
}

\begin{frame}
    \frametitle{一个程序是怎么执行的?}
    \begin{block}{Question}
        \begin{enumerate}
        \item<1-> 如果我想发布一个压缩的可执行文件,要怎么做?
            \begin{itemize}
            \item 实际上有现成的工具了,UPX。
            \end{itemize}
        \item<2-> 全局变量在哪里构造?
            \begin{itemize}
            \item C的全局变量不需要构造,只需要装载。
            \item 那么C++的全局变量呢?
            \end{itemize}
        \end{enumerate}
    \end{block}
\end{frame}

\begin{frame}[fragile]
    \frametitle{全局变量的构造与析构}
    \scriptsize
    \begin{block}{~}
        \begin{cppcode}
        #include <cstdio>
        struct S {
            int a;
            S(int x) : a(x) {printf("init %d\n", a);}
            ~S() {printf("deinit: %d\n", a);}
        };
        S s(1);
        void print() {printf("Hello: %d\n", s.a);}
        int main() {print();return 0;}
        \end{cppcode}
    \end{block}
\end{frame}

\begin{frame}[fragile]
    \frametitle{全局变量的构造与析构}
    \scriptsize
    \begin{onlyenv}<1,2>
        \begin{block}{.init\_array}
            \begin{minted}[autogobble]{text}
                Contents of section .init_array:
                3d98 60110000 00000000 ff110000 00000000  `...............
            \end{minted}
        \end{block}
    \end{onlyenv}
    \begin{onlyenv}<2,3>
        \begin{block}{\_GLOBAL\_\_sub\_I\_s}
            \begin{minted}[autogobble]{text}
                00000000000011ff <_GLOBAL__sub_I_s>:
                11ff: endbr64
                1203: push   %rbp
                1204: mov    %rsp,%rbp
                1207: mov    $0xffff,%esi
                120c: mov    $0x1,%edi
                1211: call   11a4 <__static_initialization_and_destruction_0(int, int)>
                1216: pop    %rbp
                1217: ret
            \end{minted}
        \end{block}
    \end{onlyenv}
    \begin{onlyenv}<4>
        \begin{block}{\_\_static\_initialization\_and\_destruction\_0}
            \begin{minted}[autogobble]{text}
                00000000000011a4 <__static_initialization_and_destruction_0(int, int)>:
                11a4: endbr64
                ...
                11b6: cmpl   $0x1,-0x4(%rbp)
                11ba: jne    11fc <__static_initialization_and_destruction_0(int, int)+0x58>
                11bc: cmpl   $0xffff,-0x8(%rbp)
                11c3: jne    11fc <__static_initialization_and_destruction_0(int, int)+0x58>
                11c5: mov    $0x1,%esi
                11ca: lea    0x2e4b(%rip),%rax        # 401c <s>
                11d1: mov    %rax,%rdi
                11d4: call   1218 <S::S(int)>
                11d9: lea    0x2e28(%rip),%rax        # 4008 <__dso_handle>
                11e0: mov    %rax,%rdx
                11e3: lea    0x2e32(%rip),%rax        # 401c <s>
                11ea: mov    %rax,%rsi
                11ed: lea    0x60(%rip),%rax        # 1254 <S::~S()>
                11f4: mov    %rax,%rdi
                11f7: call   1070 <__cxa_atexit@plt>
                11fc: nop
                11fd: leave
                11fe: ret
            \end{minted}
        \end{block}
    \end{onlyenv}
\end{frame}

\note[itemize]{
\item 11b6-11c5: 检查、设置标志位,防止重复构造。
    \begin{itemize}
    \item 新实现使得一个全局对象的构造和析构在多个.so里也不会在程序结束时出现double-free错误。
        但不等于说这样做是好的。
    \end{itemize}
\item 11ca-11d4: 在s上调用构造函数。
\item 11e0-11f7: 将s的析构函数注册到atexit函数的回调链上。
}

\subsection{静态库}

\begin{frame}[fragile,t]
    \frametitle{静态库}
    \scriptsize
    \begin{columns}[t]
        \begin{column}{0.33\textwidth}
            \begin{block}{a.cpp}
                \begin{cppcode}
                #include "b.hpp"
                #include <cstdio>

                void print() {
                  printf("Hello: %d\n",
                    s.a);
                }

                int main() {
                  print();
                  return 0;
                }
                \end{cppcode}
            \end{block}
        \end{column}
        \begin{column}{0.2\textwidth}
            \begin{block}{b.hpp}
                \begin{cppcode}
                #pragma once

                struct S
                {
                    int a;

                    S(int x);
                    ~S();
                };

                extern S s;
                \end{cppcode}
            \end{block}
        \end{column}
        \begin{onlyenv}<1-2>
            \begin{column}{0.33\textwidth}
                \begin{block}{b.cpp}
                    \begin{cppcode}
                    #include "b.hpp"
                    #include <cstdio>

                    S::S(int x) : a(x) {
                    printf("init %d\n",
                        a);
                    }
                    S::~S() {
                    printf("deinit: %d\n",
                        a);
                    }
                    S s(1);
                    \end{cppcode}
                \end{block}
            \end{column}
        \end{onlyenv}
        \begin{onlyenv}<3>
            \begin{column}{0.33\textwidth}
                \begin{block}{b.cpp}
                    \begin{cppcode}
                    #include "b.hpp"
                    #include <cstdio>

                    S::S(int x) : a(x) {
                    printf("init %d\n",
                        a);
                    }
                    S::~S() {
                    printf("deinit: %d\n",
                        a);
                    }
                    S s(1);
                    void xxx() {}
                    \end{cppcode}
                \end{block}
            \end{column}
        \end{onlyenv}
    \end{columns}
    \begin{onlyenv}<1>
        \begin{block}{Link with static library}
            \begin{bashcode}
                g++ -c -o b.o b.cpp
                ar r libxxx.a b.o
                g++ -c -o a.o a.cpp
                g++ -o exe a.o -lxxx -L.
            \end{bashcode}
        \end{block}
    \end{onlyenv}
    \begin{onlyenv}<2>
        \begin{block}{Link without static library}
            \begin{bashcode}
                g++ -c -o b.o b.cpp
                g++ -c -o a.o a.cpp
                g++ -o exe a.o b.o
            \end{bashcode}
        \end{block}
    \end{onlyenv}
\end{frame}

\note[itemize]{
\item 那么这两种方式的成果有没有区别呢?
}

\begin{frame}[fragile,t]
    \frametitle{静态库}
    \begin{columns}[t]
        \begin{column}{.45\textwidth}
            \begin{block}{Link with static library}
                \begin{bashcode}
                    g++ -c -o b.o b.cpp
                    ar r libxxx.a b.o
                    g++ -c -o a.o a.cpp
                    g++ -o exe a.o -lxxx -L.
                \end{bashcode}
            \end{block}
        \end{column}
        \begin{column}{.45\textwidth}
            \begin{block}{Link without static library}
                \begin{bashcode}
                    g++ -c -o b.o b.cpp
                    g++ -c -o a.o a.cpp
                    g++ -o exe a.o b.o
                \end{bashcode}
            \end{block}
        \end{column}
    \end{columns}
    \begin{block}{结果}
        \begin{consolecode}
            $ nm --demangle exe | grep xxx
            000000000000122f T xxx()
        \end{consolecode}
    \end{block}
\end{frame}

\note[itemize]{
\item 现代gcc已经将两种方式的结果改成一样了。
    这很好。
    早年的时候,静态库里不用的符号不会被链接到exe里。
    经常有人问我:为什么我把静态库改成动态库就报缺符号的错误?
    我调查之后告诉他:
    你本来就缺符号,只是恰好没有人用到。
}

\subsection{动态库}

\begin{frame}[fragile,t]
    \frametitle{动态库}
    \scriptsize
    \begin{columns}[t]
        \begin{column}{0.33\textwidth}
            \begin{block}{a.cpp}
                \begin{cppcode}
                #include "b.hpp"
                #include <cstdio>

                int main() {
                  s.print();
                  return 0;
                }
                \end{cppcode}
            \end{block}
        \end{column}
        \begin{column}{0.3\textwidth}
            \begin{block}{b.hpp}
                \begin{cppcode}
                #pragma once

                struct S
                {
                  int a;

                  S(int x);
                  ~S();
                  void print();
                };

                extern S s;
                \end{cppcode}
            \end{block}
        \end{column}
        \begin{column}{0.33\textwidth}
            \begin{block}{b.cpp}
                \begin{cppcode}
                #include "b.hpp"
                #include <cstdio>

                S::S(int x) : a(x) {}
                S::~S() {}
                void S::print() {
                  printf("s: %d\n",
                    a);
                }
                S s(1);
                \end{cppcode}
            \end{block}
        \end{column}
    \end{columns}
    \begin{onlyenv}<1>
        \begin{block}{Link with shared library}
            \begin{bashcode}
                g++ -fPIC -c -o b.o b.cpp
                g++ -shared -o libxxx.so b.o
                g++ -c -o a.o a.cpp
                g++ -Wl,--as-needed -o exe a.o -lxxx -L.
            \end{bashcode}
        \end{block}
    \end{onlyenv}
    \begin{onlyenv}<2>
        \begin{block}{然后,什么鬼?}
            \begin{consolecode}
                $ nm --demangle exe
                ...
                0000000000001060 T _start
                                 U S::print()
            \end{consolecode}
        \end{block}
    \end{onlyenv}
\end{frame}

\note[itemize]{
\item 代码里,函数调用指令接受的参数都是一个地址。
\item 而动态库里的符号,在装载之前还不知道位置。
}

\begin{frame}[fragile]
    \frametitle{Position-Independent Code, PIC}
    \scriptsize
    \begin{consolecode}
        $ objdump --disassemble --demangle exe
        ...
        Disassembly of section .text:
        0000000000001149 <main>:
            1149: endbr64
            114d: push   %rbp
            114e: mov    %rsp,%rbp
            1151: mov    0x2e90(%rip),%rax        # 3fe8 <s@Base>
            1158: mov    %rax,%rdi
            115b: call   1050 <S::print()@plt>
            1160: mov    $0x0,%eax
            1165: pop    %rbp
            1166: ret

        Disassembly of section .plt.sec:
        0000000000001050 <S::print()@plt>:
            1050: endbr64
            1054: bnd jmp *0x2f6d(%rip)        # 3fc8 <S::print()@Base>
            105b: nopl   0x0(%rax,%rax,1)
        ...
    \end{consolecode}
\end{frame}

\note[itemize]{
\item 调用者不直接调用函数S::print(),转而去调用函数的stub (S::print()@plt)。
    这样动态库的装载器就可以在运行时修改stub,填入函数的正确入口。
    (完整的过程比现在讲的要复杂一点,就不展开了)
\item 因此,符号表对动态装载很重要。
    so是不能strip -s的。
\item 值得细讲的一点。
    \begin{itemize}
    \item main里面是通过call指令去stub,但stub里是通过jmp指令跳到实际的函数。
        那么,实际函数返回的ret指令直接回到main,并不经过stub。
    \end{itemize}
    动态函数的调用经过两次跳转,那么性能如何?
    很多资料会说在5\%之内。
    我要告诉大家,日常场景下没有可以观察到的影响。
    现代CPU的流水线架构,怕条件分支,基本不怕跳转。
}

\begin{frame}
    \frametitle{动态库}
    \begin{block}{动态库有什么用?}
        \begin{itemize}
        \item 合规
        \item Foreign-Function Interface, FFI
        \item 加速整体构建
            \begin{itemize}
            \item 静态发布
            \item 动态测试
            \end{itemize}
        \end{itemize}
    \end{block}
\end{frame}

\note[itemize]{
\item 合规:著名的LGPL。
\item FFI: 比如go,如果我没记错,在.init里加载了一个golang vm。
\item 构建:实践中,链接的时间占大头;链接的时间里,磁盘IO的时间占大头。
    去链接动态库时由于不需要复制符号的内容,可以节省大量的时间。
    前面讲的阿里的那个优化的核心就是测试动态链接化。
}

\begin{frame}[plain]
    \begin{center}
        \includegraphics{pics/secret_kungfu.jpg}
    \end{center}
\end{frame}

\end{document}
